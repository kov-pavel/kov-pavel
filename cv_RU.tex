%-------------------------
% Rover Resume - Fancy Template
% Link: https://github.com/subidit/rover-resume
%------------------------

\documentclass[11pt]{article}
%\documentclass[11pt]{moderncv}

%\usepackage{helvet} % ADD HELVET BUT REMOVE HIGHLIGHTNING!!!
\renewcommand*\familydefault{\sfdefault}
\usepackage[T1, T2A]{fontenc}
%\usepackage{inter} % https://tug.org/FontCatalogue/

\usepackage[utf8]{inputenc}
\usepackage[english, russian]{babel} % ru lang

\usepackage{verbatim} % comments

\usepackage{geometry}
\geometry{
a4paper,
top=1.8cm,
bottom=1in,
left=2.5cm,
right=2.5cm
}

\setcounter{secnumdepth}{0} % remove section numbering
%\pdfgentounicode=1 % make ATS friendly

\usepackage{enumitem}
\setlist[itemize]{
    noitemsep,
    left=0pt..1.5em
}
\setlist[description]{itemsep=0pt}
\setlist[enumerate]{align=left}

% PAINTS
\usepackage[dvipsnames]{xcolor}
% \usepackage[dvipsnames, svgnames, x11names]{xcolor} 
% \usepackage[dvipsnames]{xcolor} % xcolor.pdf Sec.4 Colors by Name
\colorlet{icnclr}{gray}
% \colorlet [⟨type⟩]{⟨name⟩}[⟨num model⟩]{⟨color ⟩}
% \definecolor[⟨type⟩]{⟨name⟩}{⟨model-list⟩}{⟨spec-list⟩}

% PARTIOTIONING & SPACING
\usepackage{titlesec}
% \titlespacing{command}{left spacing}{before spacing}{after spacing}[right]
% \titlespacing{\section}{0pt}{*3}{*1}
\titlespacing{\subsection}{0pt}{*0}{*0}
\titlespacing{\subsubsection}{0pt}{*0}{*0}
% \titleformat{<command>}[<shape>]{<format>}{<label>}{<sec>}{<before-code>}[<after-code>]  
%\titleformat{\section}{\color{Sepia}\large\fontseries{black}\selectfont\uppercase}{}{}{\ruleafter}[\global\RemVStrue] % A PROBLEM!
%\usepackage[koi8-r]{inputenc}
\titleformat{\section}{\color{Sepia}\large\fontseries{black}\selectfont\uppercase}{}{}{\ruleafter}[\global\RemVStrue]
\titleformat{\subsection}{\large\fontseries{semibold}\selectfont}{}{}{\rvs}
\titleformat{\subsubsection}{\large\fontseries{medium}\selectfont}{}{}{}

% HORIZONTAL LINE AFTER EACH TITLE
\usepackage{xhfill} 
\newcommand\ruleafter[1]{#1~\xrfill[.5ex]{1pt}[gray]} % add rule after title in .5 x-height 

% REMOVE VSPACE BETWEEN \section & \subsection
\newif\ifRemVS
\newcommand{\rvs}{
    \ifRemVS
        \vspace{-1.5ex}
    \fi
    \global\RemVSfalse
}

% ICONS
\usepackage{fontawesome5}

% DOC BROWSER TITLE
\usepackage[bookmarks=false]{hyperref} % [imp!]
\hypersetup{ % https://en.wikibooks.org/wiki/LaTeX/Hyperlinks
    colorlinks=true,
    urlcolor=Sepia,
    pdftitle={Павел Коваленко Резюме},
}

% PAGE NUMBER AT THE BOTTOM OF DOC
\usepackage[page]{totalcount}
\usepackage{fancyhdr}
\pagestyle{fancy}
\renewcommand{\headrulewidth}{0pt}	
\fancyhf{}							
\cfoot{\color{darkgray} Павел Коваленко -- Страница \thepage{} из \totalpages}

\begin{document}

%== HEADER ==%
\begin{center}
    {\fontsize{36}{36}\selectfont ПАВЕЛ КОВАЛЕНКО} \\ \bigskip
    {\fontsize{25}{25}\selectfont Middle Java Software Developer} \\ \bigskip
    {\color{icnclr}\faEnvelope[email]} \href{mailto:kovalenkopavelb@gmail.com}{kovalenkopavelb@gmail.com} $|$ 
    {\color{icnclr}\faIcon{mobile-alt}} 89214332531 $|$
    {\color{icnclr}\faMapMarker} Санкт-Петербург
\end{center}

\section{EDUCATION}
%==============
\subsection{\href{https://www.spbstu.ru/}{\underline{СПбПУ}} $|$ {\normalfont\textit{\href{https://www.spbstu.ru/structure/graduate_school_software_engineering/}{\underline{Программная инженерия}}}} \hfill 2020 - 2024}
\begin{itemize}
    \item \underline{\textbf{Стек}: Java, PostgreSQL, Spring (Boot, Web, Data, Security), Docker (+ Compose), Python}
\end{itemize}

\subsection{\href{https://polis.vk.company/}{\underline{Технополис} $|$ {\normalfont\textit{\underline{Java-разработчик высоконагруженных приложений}}}} \hfill 2021 - 2023}
\begin{itemize}
    \item \underline{\textbf{Стек}: Java, Cassandra, Spark, Kafka, Airflow, Hadoop, Scala, JUnit, Kotlin}
\end{itemize}

\section{EXPERIENCE (TOTAL - 1 year 3 months)}
%===================
\subsection{\href{https://dzen.ru}{\underline{Дзен}} $|$ {\normalfont\textit{Java Developer Intern}} \hfill Март 2023 -- Июль 2023 (4 месяца)}
\subsubsection{\underline{\textbf{Стек}: Java, ClickHouse, MongoDB, YDB, gRPC, Yandex.Monitoring, Python, TeamCity, Jira}}
\subsubsection{Я создал \textbf{систему мониторинга и оповещения} для контроля состояния и производительности сервиса, запустил несколько \textbf{A/B тестов}, которые позволили найти неиспользуемую функциональность, создал \textbf{автоматические Python-скрипты} для рефакторинга большого объёма кода, а также \textbf{написал документацию} о системах мониторинга Дзена.}
\begin{itemize}
\end{itemize}

\subsection{\href{https://github.com/Sanerins/tamtam-one-coffee-bot}{\underline{Технополис $|$ {\normalfont\textit{Java Developer}}}} \hfill Февраль 2023 -- Июнь 2023 (5 месяцев)}
\subsubsection{\underline{\textbf{Стек}: Java, Spring (Boot), PostgreSQL}}
\subsubsection{В соответствие с бизнес-кейсом компании \href{https://ok.ru/}{\underline{<<Одноклассники>>}} в команде разрабатывал бота для знакомств для улучшения социальной коммуникации внутри компании. Я создал \textbf{интерфейсы и классы для взаимодействия с PostgreSQL}, а также добавил \textbf{сохранение истории диалогов пользователей} для улучшения поиска собеседников.}
\begin{itemize}
\end{itemize}

\subsection{\href{https://github.com/PaaavelZ/FPA-pybot}{\underline{СПбПУ $|$ {\normalfont\textit{Python Developer}}}} \hfill Февраль 2022 -- Июнь 2022 (5 месяцев)}
\subsubsection{\underline{\textbf{Стек}: Python, SQLite, Docker, Telegram API}}
\subsubsection{В рамках бизнес-задачи компании \href{https://www.wbc-c.ru/}{\underline{<<Белорусская косметика>>}} я разрабатывал в команде бота для защиты медиа-контента от плагиата. Я создал \textbf{интерфейсы и классы для взаимодействия с SQLite}, добавил \textbf{архивную обработку фотографий} и написал документацию к проекту.}
\begin{itemize}
\end{itemize}

\subsection{\href{https://github.com/martyn-fanclub/tracking-system}{\underline{Завод на базе СПбПУ $|$ {\normalfont\textit{Java Developer}}}} \hfill Октябрь 2021 -- Январь 2022 (4 месяца)}
\subsubsection{\underline{\textbf{Стек}: Java, Spring (Boot, Web, Data JPA, Security)}}
\subsubsection{Я разрабатывал в команде \textbf{систему отслеживания прогресса работы сотрудников предприятия на станках}. Я реализовал \textbf{глобальный таймер} для отслеживания текущего времени и имплементировал несколько \textbf{эндпоинтов} на серверной части.}
\begin{itemize}
\end{itemize}

\section{Projects}
%=================
\subsection{\href{https://github.com/polis-vk/2022-highload-dht/tree/main/src/main/java/ok/dht/test/kovalenko}{\underline{Технополис $|$ {\normalfont\textit{Java Developer}}}} \hfill Февраль 2022 -- Декабрь 2022}
\subsubsection{\underline{\textbf{Стек}: Java, Cassandra, HTTP REST API, JUnit}}
\subsubsection{Я разрабатывал \textbf{свою NoSQL базу данных}, для которой Cassandra была взята за прототип. На первом этапе реализовал \textbf{Persistence}, \textbf{внутренние range-запросы} и \textbf{Compaction}. На этапе сетевого взаимодействия я имплементировал \textbf{HTTP REST API}, \textbf{асинхронный сервер}, \textbf{шардирование}, \textbf{репликацию}, \textbf{асинхронное взаимодействие} и \textbf{сетевые range-запросы}.}
\begin{itemize}
\end{itemize}

\end{document}
